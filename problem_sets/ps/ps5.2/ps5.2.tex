\documentclass[
  % all of the below options are optional and can be left out
  % course name (default: 2IL50 Data Structures)
  course = {{MATH102 Calculus II}},
  % quartile (default: 3)
  quartile = {{2}},
  % assignment number/name (default: 1)
  assignment = {{Section: 5.2}},
  % student name (default: Some One)
  %name = {{Some One ; Other Person}},
  % student number, NOT S-number (default: 0123456)
  %studentnumber = {{0123456 ; 0314159}},
  % student email (default: s.one@student.tue.nl)
  %email = {{s.one@student.tue.nl ; o.person@student.tue.nl}},
  % first exercise number (default: 1)
  firstexercise = 1,
  term = 203
]{../class/aga-homework}
\usepackage{graphicx}
\usepackage{advdate}
\usepackage{amsmath}
\usepackage[shortlabels]{enumitem}


\begin{document}
%\noindent {\hfill \vspace{-2\baselineskip} {\footnotesize\bf  Due \AdvanceDate[14]\today  \; midnight} \vspace{3mm}}
%
%\begin{center}
%  {\large
%  \begin{tabular}{|l|l|}
%    \hline
%    % after \\: \hline or \cline{col1-col2} \cline{col3-col4} ...
%    Group Number: & \hspace{3in} \\
%     &  \\ \hline
%    Members: &  \\
%        &  \\ \cline{2-2}
%        &  \\
%        &  \\ \cline{2-2}
%        &  \\
%        &  \\ \cline{2-2}
%        &  \\
%        &  \\ \cline{2-2}
%        &  \\
%        &  \\ \cline{2-2}
%        &  \\
%        &  \\ \cline{2-2}
%        &  \\
%        &  \\ \cline{2-2}
%        &  \\
%        &  \\ \cline{2-2}
%        &  \\
%        &  \\ \cline{2-2}
%        &  \\
%        &  \\ \cline{2-2}
%        &  \\
%        &  \\ \cline{2-2}
%        &  \\
%        &  \\ \cline{2-2}
%        &  \\
%        &  \\ \cline{2-2}
%        &  \\
%        &  \\ \cline{2-2}
%        &  \\
%        &  \\ \cline{2-2}
%        &  \\
%        &  \\
%    \hline
%  \end{tabular}
%  }
%\end{center}
%
%
%\newpage

\problem  If $R_n$ is the Riemann sum for $f(x)= 4 + \frac{x^2}{8}, 0 \leq x \leq 4$
with $n$ subintervals and taking sample points to be the right end points, then $R_n =$


\newpage

\problem $\lim_{n\to \infty}\sum_{i=1}^n\frac{1}{n}\cos\left(1+\frac{i}{n}\right)^2=$

\begin{enumerate}[(A)]
  \item $\int_{1}^{2}\cos(1+x^2)dx$.
  \item $\int_{1}^{2}\cos(x^2)dx$.
  \item $\int_{1}^{2}\cos^2(x)dx$.
  \item $\int_{0}^{1}\cos(x^2)dx$.
  \item $\int_{0}^{1}\cos(1+x^2)dx$.
\end{enumerate}

\newpage


\problem
\begin{minipage}[t]{0.6\textwidth}
\vspace{0pt}
  In the figure shown, regions $A$ and $B$ are bounded by the graph of a function $f$ and the $x$-axis. If the area of region $A$ is $\frac{1}{6}$ and the area of the region $B$ is $\frac{3}{8}$, then
  \[
  \int_0^4f(x) dx + \int_{0}^{4}|f(x)| dx =
  \]
\end{minipage}
\begin{minipage}[r]{0.4\textwidth}
	\includegraphics[width=\textwidth]{../figures/ps2fig1.png}
\end{minipage}

\newpage


\problem If $\int_{-5}^7f(x)dx=-17, \int_{-5}^{11}f(x)dx=32$, and $\int_{8}^7f(x)dx=5$, then $\int_{11}^8f(x)dx=$.

\newpage

\problem If ${\displaystyle f(x) = \left\{\begin{array}{ll}
         -x; & -4\leq x <0 \\
         \\
         \sqrt{4-x^2}; & 0\leq x \leq 2 \\
       \end{array}\right.}$, then the value of the integral
       $ \displaystyle
       \int_{-4}^{2} f(x) dx
       $
       by interpreting in terms of area(s) is.


\newpage
\problem Write the limit as an integral (do not evaluate) \[\displaystyle \lim_{n\to \infty} \sum_{i=1}^{n}\left[1+\sin\left(1+\frac{i}{n}\right)\right]\frac{2}{n}=
\]

\newpage

\problem $\displaystyle \lim_{n\to\infty} \frac{2}{n^4}(1+8+27+\cdots+n^3)=$
\newpage




\problem If $f$ is continuous function and
 \[ 2\leq f(x) \leq 5 \quad \text{for} \quad 3\leq x \leq 9,\]
then ONE of the following statements is **FALSE**

\begin{enumerate}[(A)]
    \item $\int_{3}^{9}\left|f(x)\right|dx\geq 12$

    \item $\int_{3}^{9}\left(3-f(x)\right)dx\geq -12$

    \item $\int_{3}^{9}\left(1-\left|f(x)\right|\right)dx\geq -10$

    \item $\int_{3}^{9}-2f(x)dx\leq -24$

    \item $\int_{3}^{9}\left(f(x)\right)^2dx\geq 24$
\end{enumerate}



\end{document} 