\documentclass[
  % all of the below options are optional and can be left out
  % course name (default: 2IL50 Data Structures)
  course = {{MATH102 Calculus II}},
  % quartile (default: 3)
  quartile = {{2}},
  % assignment number/name (default: 1)
  assignment = {{Sections 11.8}},%Monday, %Wednesday,
  topic = {{Power Series}},
  % student name (default: Some One)
  %name = {{Some One ; Other Person}},
  % student number, NOT S-number (default: 0123456)
  %studentnumber = {{0123456 ; 0314159}},
  % student email (default: s.one@student.tue.nl)
  %email = {{s.one@student.tue.nl ; o.person@student.tue.nl}},
  % first exercise number (default: 1)
  firstexercise = 1,
  term = 203
]{../class/aga-homework}

\usepackage{graphicx}
\usepackage{advdate}
\usepackage{amsmath}
\usepackage[shortlabels]{enumitem}
\usepackage{afterpage}

\begin{document}
%\noindent {\hfill \vspace{-2\baselineskip} {\footnotesize\bf  Due \AdvanceDate[14]\today  \; midnight} \vspace{3mm}}
%
%\begin{center}
%  {\large
%  \emph{Please indicate the members who are present. Also indicate the group coordinator.}
%  \begin{tabular}{|l|l|}
%    \hline
%    % after \\: \hline or \cline{col1-col2} \cline{col3-col4} ...
%    Group Number: & \hspace{3in} \\
%     &  \\ \hline
%    Members: &  \\
%        &  \\ \cline{2-2}
%        &  \\
%        &  \\ \cline{2-2}
%        &  \\
%        &  \\ \cline{2-2}
%        &  \\
%        &  \\ \cline{2-2}
%        &  \\
%        &  \\ \cline{2-2}
%        &  \\
%        &  \\ \cline{2-2}
%        &  \\
%        &  \\ \cline{2-2}
%        &  \\
%        &  \\ \cline{2-2}
%        &  \\
%        &  \\ \cline{2-2}
%        &  \\
%        &  \\ \cline{2-2}
%        &  \\
%        &  \\ \cline{2-2}
%
%    \hline
%  \end{tabular}
%  }
%\end{center}
%
%\fbox{
%\begin{minipage}{\textwidth}
%
%\mbox{}
%
%\vspace{1mm}
%If $\displaystyle s=\sum (-1)^{n-1}b_n$ where $b_n>0$ is the sum of an alternating series that satisfies
%\[
%\displaystyle
%\text{(i)} \; b_{n+1}\leq b_n \quad \text{and} \quad \text{(ii)} \; \lim_{n\to \infty}b_n = 0
%\]
%then
%\[
%|R_n|=|s-s_n|\leq b_{n+1}
%\]
%\vspace{2mm}
%\hrule
%\vspace{2mm}
%A series $\displaystyle \sum a_n$ is called {\bf \emph{absolutely convergent}} if and only if the series of the absolute values $\displaystyle \sum |a_n|$ is convergent.
%\vspace{2mm}
%\hrule
%\vspace{2mm}
%A series $\displaystyle \sum a_n$ is called {\bf \emph{conditionally convergent}} if $\displaystyle \sum a_n$ converges but $\displaystyle \sum |a_n|$ diverges.
%
%\end{minipage}
%}

\newpage

\problem Find the radius of convergence and interval of convergence of the power series $\displaystyle \sum_{n=1}^{\infty}\frac{x^{n}}{n3^n}$.

\newpage

\problem Find the radius of convergence and interval of convergence of the power series $\displaystyle \sum_{n=0}^{\infty}\frac{(x-2)^{n}}{n^2+1}$.

\newpage

\problem Find the radius of convergence and interval of convergence of the power series $\displaystyle \sum_{n=1}^{\infty}\frac{(x-2)^{n}}{n^n}$.

\newpage

\problem Find the radius of convergence and interval of convergence of the power series $\displaystyle \sum_{n=1}^{\infty}n!(2x-1)^n $.


\newpage
\afterpage{\null\newpage}

\afterpage{\null\newpage}

\afterpage{\null\newpage}

\end{document} 