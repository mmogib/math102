\documentclass[
  % all of the below options are optional and can be left out
  % course name (default: 2IL50 Data Structures)
  course = {{MATH102 Calculus II}},
  % quartile (default: 3)
  quartile = {{2}},
  % assignment number/name (default: 1)
  assignment = {{Sections 11.3}},%Monday, %Wednesday,
  topic = {{The Integral Test}},
  % student name (default: Some One)
  %name = {{Some One ; Other Person}},
  % student number, NOT S-number (default: 0123456)
  %studentnumber = {{0123456 ; 0314159}},
  % student email (default: s.one@student.tue.nl)
  %email = {{s.one@student.tue.nl ; o.person@student.tue.nl}},
  % first exercise number (default: 1)
  firstexercise = 1,
  term = 203
]{../class/aga-homework}

\usepackage{graphicx}
\usepackage{advdate}
\usepackage{amsmath}
\usepackage[shortlabels]{enumitem}
\usepackage{afterpage}

\begin{document}
%\noindent {\hfill \vspace{-2\baselineskip} {\footnotesize\bf  Due \AdvanceDate[14]\today  \; midnight} \vspace{3mm}}
%
%\begin{center}
%  {\large
%  \emph{Please indicate the members who are present. Also indicate the group coordinator.}
%  \begin{tabular}{|l|l|}
%    \hline
%    % after \\: \hline or \cline{col1-col2} \cline{col3-col4} ...
%    Group Number: & \hspace{3in} \\
%     &  \\ \hline
%    Members: &  \\
%        &  \\ \cline{2-2}
%        &  \\
%        &  \\ \cline{2-2}
%        &  \\
%        &  \\ \cline{2-2}
%        &  \\
%        &  \\ \cline{2-2}
%        &  \\
%        &  \\ \cline{2-2}
%        &  \\
%        &  \\ \cline{2-2}
%        &  \\
%        &  \\ \cline{2-2}
%        &  \\
%        &  \\ \cline{2-2}
%        &  \\
%        &  \\ \cline{2-2}
%        &  \\
%        &  \\ \cline{2-2}
%        &  \\
%        &  \\ \cline{2-2}
%
%    \hline
%  \end{tabular}
%  }
%\end{center}

\fbox{
\begin{minipage}{\textwidth}

\mbox{}

\vspace{1mm}
The $p-$ series
\[
\displaystyle
\sum_{n=1}^{\infty}\frac{1}{n^p}
\]

converges if $p>1$ and diverges if $p\leq 1$.

\vspace{3mm}
\hrule

\vspace{3mm}
Suppose $f(k)=a_k$, where $f$ is a continuous, positive, decreasing function for $x\geq n$ and \linebreak $\displaystyle \sum_{n=1}^{\infty}a_n$  is convergent.
If $R_n=s-s_n$, then
\[
\int_{n+1}^{\infty}f(x) dx\leq R_n \leq \int_{n }^{\infty}f(x) dx
\]

\end{minipage}
}

\newpage

\problem Determine whether the series is convergent or divergent
$\displaystyle \sum_{n=1}^{\infty}\frac{1}{n^{\sqrt{2}}}$

\newpage

\problem Determine whether the series is convergent or divergent
$\displaystyle \sum_{n=1}^{\infty}\frac{\sqrt{n}+4}{n^{2}}$

\newpage

\problem Determine whether the series is convergent or divergent
$\displaystyle \sum_{n=1}^{\infty}\frac{\sqrt{n}}{1+n^{3/2}}$

\newpage

\problem Determine whether the series is convergent or divergent
$\displaystyle \sum_{n=2}^{\infty}\frac{\ln n}{n^{2}}$

\newpage

\problem  Explain why the Integral Test can’t be used to determine whether the series is convergent $\displaystyle \sum_{n=1}^{\infty}\frac{\cos \pi n}{\sqrt{n}}$.

\newpage

\problem Find the values of $p$ for which the series is convergent. \\
$$\displaystyle \sum_{n=1}^{\infty}n(1+n^2)^p$$

\newpage

\problem Given that $\displaystyle \sum_{n=1}^{\infty}\frac{1}{n^2}=\frac{\pi^2}{6}\qquad \quad$
find the sum of
$\displaystyle \sum_{n=3}^{\infty}\frac{1}{(n+1)^2}$

\newpage

\problem How many terms of the series $\displaystyle \sum_{n=1}^{\infty}\frac{1}{n(\ln n)^2}$ would you need to add to find its sum to within $0.01$?
\newpage

\afterpage{\null\newpage}

\afterpage{\null\newpage}

\afterpage{\null\newpage}

\end{document} 